\documentclass{article}

% Packages for differential geometry
\usepackage{amsmath, amssymb, amsthm}
\usepackage{tikz-cd} % For commutative diagrams

\theoremstyle{definition}
\newtheorem{definition}{Definition}[section]
\newtheorem{theorem}[definition]{Theorem}
\newtheorem{example}[definition]{Example}

\numberwithin{equation}{section}

% Title and author information
\title{An Introduction to Differential Geometry}
\author{AI-generated content by OpenAI's GPT-3}
\date{\today}

\begin{document}
\maketitle

\begin{abstract}
This article on differential geometry is generated by OpenAI's language model, GPT-3. While efforts have been made to provide accurate and relevant information, it is essential to acknowledge that AI-generated content may contain errors, inaccuracies, or lack complete human understanding. Readers should exercise caution and critically evaluate the information presented. It is recommended to consult authoritative sources and subject matter experts for further verification and clarification. OpenAI and the author do not guarantee the accuracy or reliability of the information provided in this article. Use this content responsibly and at your own discretion.
\end{abstract}

\section{Introduction}
Differential geometry is a branch of mathematics that studies smooth and curved spaces using calculus and algebraic techniques. It provides a powerful framework to study geometric objects, such as curves and surfaces, and their properties. This paper serves as an introduction to some of the central ideas and results in differential geometry.

\section{Smooth Manifolds}
In this section, we introduce the concept of smooth manifolds and define differentiable maps between them.

\begin{definition}[Smooth Manifold]
A \textit{smooth manifold} of dimension $n$ is a Hausdorff topological space $M$ equipped with an atlas $\mathcal{A} = \{(U_\alpha, \phi_\alpha)\}$, where each $U_\alpha \subseteq M$ is open and $\phi_\alpha: U_\alpha \to \mathbb{R}^n$ is a homeomorphism. The maps $\phi_\alpha$ are called \textit{smooth charts}, and their compositions $\phi_\alpha \circ \phi_\beta^{-1}$ on $U_\alpha \cap U_\beta$ are smooth.
\end{definition}

\begin{theorem}
Let $M$ be a smooth manifold. Then, for any smooth chart $(U, \phi)$ on $M$, the transition maps $\phi \circ \psi^{-1}: \phi(U \cap V) \to \psi(U \cap V)$ are smooth for all other charts $(V, \psi)$ in the atlas $\mathcal{A}$.
\end{theorem}

\begin{proof}
Let $(U, \phi)$ and $(V, \psi)$ be two smooth charts in $\mathcal{A}$, and let $p \in U \cap V$. Without loss of generality, assume $p = \psi^{-1}(x_0) \in V$, where $x_0 \in \mathbb{R}^n$. Since $\phi$ and $\psi$ are homeomorphisms, $U \cap V$ is an open set in both $\mathbb{R}^n$ and $M$.

Now, consider the composition $\phi \circ \psi^{-1}: \phi(U \cap V) \to \psi(U \cap V)$. For any $x \in \phi(U \cap V)$, we have $x = \phi(p')$ for some $p' \in U \cap V$. Thus,
\[\begin{aligned}
(\phi \circ \psi^{-1})(x) &= \phi(\psi^{-1}(x)) \\
&= \phi(p') \\
&= \phi \circ \psi^{-1}(p') \\
&= \phi \circ \psi^{-1} \circ \psi \circ \psi^{-1}(p') \\
&= (\phi \circ \psi^{-1} \circ \psi)(x_0).
\end{aligned}\]
Since $\phi \circ \psi^{-1} \circ \psi$ is the composition of smooth functions, it is itself smooth. Therefore, $\phi \circ \psi^{-1}$ is smooth on $\phi(U \cap V)$, and the theorem follows.
\end{proof}

\begin{definition}[Differentiable Map]
Let $M$ and $N$ be smooth manifolds. A map $f: M \to N$ is called \textit{differentiable} (or smooth) if, for every smooth chart $(U, \phi)$ on $M$ and smooth chart $(V, \psi)$ on $N$, the composition $\psi^{-1} \circ f \circ \phi$ is smooth as a map from $\phi(U \cap f^{-1}(V))$ to $\psi(V)$.
\end{definition}

\begin{theorem}
The composition of differentiable maps is differentiable. That is, if $f: M \to N$ and $g: N \to P$ are differentiable maps between smooth manifolds, then $g \circ f: M \to P$ is also differentiable.
\end{theorem}

\begin{proof}
Let $(U, \phi)$ be a smooth chart on $M$, $(V, \psi)$ be a smooth chart on $N$, and $(W, \chi)$ be a smooth chart on $P$. Since $f$ and $g$ are differentiable, we have smooth compositions $\psi^{-1} \circ f \circ \phi$ and $\chi^{-1} \circ g \circ \psi$, which are both smooth maps between open subsets of $\mathbb{R}^n$. Therefore, their composition $\chi^{-1} \circ g \circ \psi \circ \psi^{-1} \circ f \circ \phi = \chi^{-1} \circ (g \circ f) \circ \phi$ is also smooth. Hence, $g \circ f$ is differentiable, as claimed.
\end{proof}

\section{Tangent Spaces}
The tangent space $T_pM$ at a point $p \in M$ is the set of all tangent vectors to $M$ at $p$. We can identify $T_pM$ with the vector space $\mathbb{R}^n$ through a choice of coordinates. A smooth curve $\gamma: I \to M$ with $\gamma(0) = p$ is given by $\gamma(t) = (x^1(t), x^2(t), \ldots, x^n(t))$ in local coordinates, and its tangent vector at $p$ is given by
$$\gamma'(0) = \left(\frac{dx^1}{dt}(0), \frac{dx^2}{dt}(0), \ldots, \frac{dx^n}{dt}(0)\right).$$
\section{Curves and Surfaces}
A \textit{curve} in a smooth manifold $M$ is a smooth map $\gamma: I \to M$, where $I$ is an interval in $\mathbb{R}$. We say that a curve is \textit{regular} if its velocity vector $\gamma'(t)$ is nonzero for all $t \in I$. The \textit{velocity vector} of a regular curve $\gamma$ at $p$ is the tangent vector $\gamma'(0) \in T_pM$.

\begin{example}[Parametric Curves in $\mathbb{R}^3$]
Consider the parametric curve $\gamma: \mathbb{R} \to \mathbb{R}^3$ defined by
\[\gamma(t) = (t, t^2, \sin(t)).\]
The velocity vector of this curve is given by
\begin{equation}\label{eq:parametric_curve}
\gamma'(t) = (1, 2t, \cos(t)).
\end{equation}
The tangent vector at $t = 0$ is $\gamma'(0) = (1, 0, 1)$. \cite{do-carmo}
\end{example}

\begin{example}[Circle in $\mathbb{R}^2$]
Let $S^1$ be the unit circle in $\mathbb{R}^2$ defined by
\[S^1 = \{(x, y) \in \mathbb{R}^2 \mid x^2 + y^2 = 1\}.\]
We can parametrize $S^1$ using the angle $\theta$ as $\gamma(\theta) = (\cos(\theta), \sin(\theta))$. The velocity vector is given by
\begin{equation}\label{eq:circle}
\gamma'(\theta) = (-\sin(\theta), \cos(\theta)).
\end{equation}
The tangent vector at $\theta = 0$ is $\gamma'(0) = (0, 1)$.
\end{example}

\begin{example}[Helix in $\mathbb{R}^3$]
Let $S^1$ be the unit circle in $\mathbb{R}^2$ as before. We can form a helix in $\mathbb{R}^3$ by rotating $S^1$ along the $z$-axis. Parametrize the helix using the angle $\theta$ as
\[\gamma(\theta) = (\cos(\theta), \sin(\theta), \theta).\]
The velocity vector is
\begin{equation}\label{eq:helix}
\gamma'(\theta) = (-\sin(\theta), \cos(\theta), 1).
\end{equation}
The tangent vector at $\theta = 0$ is $\gamma'(0) = (0, 1, 1)$.
\end{example}

\begin{example}[Sphere in $\mathbb{R}^3$]
Consider the sphere $S^2$ of radius $r$ in $\mathbb{R}^3$ centered at the origin, given by
\[S^2 = \{(x, y, z) \in \mathbb{R}^3 \mid x^2 + y^2 + z^2 = r^2\}.\]
To parametrize $S^2$, we can use spherical coordinates:
\[\gamma(\theta, \phi) = (r\sin(\phi)\cos(\theta), r\sin(\phi)\sin(\theta), r\cos(\phi)),\]
where $0 \leq \theta \leq 2\pi$ and $0 \leq \phi \leq \pi$. The velocity vectors are
\begin{align}
\gamma_{\theta}'(\theta, \phi) &= (-r\sin(\phi)\sin(\theta), r\sin(\phi)\cos(\theta), 0), \label{eq:sphere_theta}\\
\gamma_{\phi}'(\theta, \phi) &= (r\cos(\phi)\cos(\theta), r\cos(\phi)\sin(\theta), -r\sin(\phi)). \label{eq:sphere_phi}
\end{align}
The tangent vectors at the North Pole $(\theta = 0, \phi = 0)$ are $\gamma_{\theta}'(0, 0) = (0, r, 0)$ and $\gamma_{\phi}'(0, 0) = (r, 0, 0)$.
\end{example}

A \textit{surface} in a smooth manifold $M$ is a subset $S \subseteq M$ such that for every point $p \in S$, there exists a neighborhood $U$ of $p$ and a smooth map $\phi: U \to \mathbb{R}^2$ such that $\phi(U \cap S) = \{(u, v) \in \mathbb{R}^2 \mid u \geq 0\}$.

\section{First Fundamental Form}
The \textit{first fundamental form} is a fundamental concept in the study of surfaces. It encodes the intrinsic geometry of a surface, capturing information about the lengths and angles of curves on the surface.

\begin{definition}
Let $S$ be a smooth surface in $\mathbb{R}^3$, and let $\mathbf{X}: U \subset \mathbb{R}^2 \to \mathbb{R}^3$ be a smooth parametrization of $S$. The first fundamental form is defined as
\[I = E\,du^2 + 2F\,du\,dv + G\,dv^2,\]
where $E = \mathbf{X}_u \cdot \mathbf{X}_u$, $F = \mathbf{X}_u \cdot \mathbf{X}_v$, and $G = \mathbf{X}_v \cdot \mathbf{X}_v$ are the coefficients of the first fundamental form, and $du$ and $dv$ are the differentials of the parameters $u$ and $v$, respectively.
\end{definition}

The coefficients $E$, $F$, and $G$ of the first fundamental form depend on the choice of parametrization of the surface. However, the invariants such as the Gaussian curvature and the total curvature remain invariant under isometries \cite{do-carmo}.

\subsection{Calculations}
To illustrate the first fundamental form, let's consider the parametric curve
\[\gamma(u, v) = (u, u^2, v).\]
The coefficients of the first fundamental form are calculated as follows:

\begin{align*}
E &= \mathbf{X}_u \cdot \mathbf{X}_u = (1, 2u, 0) \cdot (1, 2u, 0) = 1 + 4u^2, \\
F &= \mathbf{X}_u \cdot \mathbf{X}_v = (1, 2u, 0) \cdot (0, 0, 1) = 0, \\
G &= \mathbf{X}_v \cdot \mathbf{X}_v = (0, 0, 1) \cdot (0, 0, 1) = 1.
\end{align*}

Thus, the first fundamental form for this parametric curve is
\[I = (1 + 4u^2)\,du^2 + dv^2.\]

\section{Intrinsic vs. Extrinsic Geometry}
Intrinsic properties of geometric objects are those that are independent of the embedding in higher-dimensional spaces. For example, the length of a curve, the curvature of a surface, and geodesics are all intrinsic properties.

Extrinsic properties, on the other hand, depend on the embedding. For instance, the normal vector and the second fundamental form of a surface in $\mathbb{R}^3$ are extrinsic properties.

\section{Theorema Egregium}
The \textit{Theorema Egregium} (Remarkable Theorem) is a fundamental result in differential geometry, discovered by Carl Friedrich Gauss \cite{gauss}. It establishes an intrinsic curvature property of surfaces and highlights the remarkable fact that Gaussian curvature is invariant under isometric mappings.

\begin{theorem}[Theorema Egregium]
Let $S$ be a smooth surface in $\mathbb{R}^3$ with a smooth metric $g$, and let $\bar{S}$ be another surface obtained by isometrically mapping $S$ onto itself. Then, the Gaussian curvature $K$ of $S$ is equal to the Gaussian curvature $\bar{K}$ of $\bar{S}$ at corresponding points.
\end{theorem}

\begin{proof}
Let $\mathbf{X} = (x, y, z)$ be a coordinate system in $\mathbb{R}^3$ and $(u, v)$ be a coordinate system on $S$. We assume that $g$ is the metric induced on $S$ by the Euclidean metric $\mathbf{G} = d\mathbf{X} \cdot d\mathbf{X}$. Similarly, let $\bar{\mathbf{X}} = (\bar{x}, \bar{y}, \bar{z})$ be another coordinate system in $\mathbb{R}^3$ and $(\bar{u}, \bar{v})$ be a coordinate system on $\bar{S}$, induced by the Euclidean metric $\bar{\mathbf{G}} = d\bar{\mathbf{X}} \cdot d\bar{\mathbf{X}}$.

Let $\mathbf{E}, \mathbf{F}, \mathbf{G}$ be the coefficients of the first fundamental form for $S$, and $\bar{\mathbf{E}}, \bar{\mathbf{F}}, \bar{\mathbf{G}}$ be the coefficients of the first fundamental form for $\bar{S}$.

We can write the first fundamental form coefficients for $S$ in terms of the coordinate system $(u, v)$ as follows:
\begin{align*}
\mathbf{E} &= \mathbf{X}_u \cdot \mathbf{X}_u, \\
\mathbf{F} &= \mathbf{X}_u \cdot \mathbf{X}_v, \\
\mathbf{G} &= \mathbf{X}_v \cdot \mathbf{X}_v,
\end{align*}
where subscripts denote partial derivatives.

Similarly, for $\bar{S}$, the first fundamental form coefficients in the coordinate system $(\bar{u}, \bar{v})$ are given by:
\begin{align*}
\bar{\mathbf{E}} &= \bar{\mathbf{X}}_{\bar{u}} \cdot \bar{\mathbf{X}}_{\bar{u}}, \\
\bar{\mathbf{F}} &= \bar{\mathbf{X}}_{\bar{u}} \cdot \bar{\mathbf{X}}_{\bar{v}}, \\
\bar{\mathbf{G}} &= \bar{\mathbf{X}}_{\bar{v}} \cdot \bar{\mathbf{X}}_{\bar{v}}.
\end{align*}

Since $S$ and $\bar{S}$ are isometric, the first fundamental forms are equal:
\begin{align}
\mathbf{E} &= \bar{\mathbf{E}}, \label{eq:egregium_1} \\
\mathbf{F} &= \bar{\mathbf{F}}, \label{eq:egregium_2} \\
\mathbf{G} &= \bar{\mathbf{G}}. \label{eq:egregium_3}
\end{align}

The Gaussian curvature $K$ of $S$ is given by the determinant of the second fundamental form divided by the determinant of the first fundamental form:
\begin{equation}\label{eq:egregium_K}
K = \frac{\mathbf{EG} - \mathbf{F}^2}{\mathbf{EG}}.
\end{equation}

Similarly, the Gaussian curvature $\bar{K}$ of $\bar{S}$ is given by:
\begin{equation}\label{eq:egregium_K_bar}
\bar{K} = \frac{\bar{\mathbf{E}}\bar{\mathbf{G}} - \bar{\mathbf{F}}^2}{\bar{\mathbf{E}}\bar{\mathbf{G}}}.
\end{equation}

Substituting Equations \eqref{eq:egregium_1}, \eqref{eq:egregium_2}, and \eqref{eq:egregium_3} into Equations \eqref{eq:egregium_K} and \eqref{eq:egregium_K_bar}, we obtain:
\begin{align*}
K &= \frac{\bar{\mathbf{E}}\bar{\mathbf{G}} - \bar{\mathbf{F}}^2}{\bar{\mathbf{E}}\bar{\mathbf{G}}}, \\
\bar{K} &= \frac{\bar{\mathbf{E}}\bar{\mathbf{G}} - \bar{\mathbf{F}}^2}{\bar{\mathbf{E}}\bar{\mathbf{G}}}.
\end{align*}

Since $K$ and $\bar{K}$ have the same expression, they are equal at corresponding points on $S$ and $\bar{S}$. This concludes the proof.
\end{proof}

\section{Applications}
Differential geometry has applications in various fields, including physics, computer graphics, and robotics \cite{do-carmo, pressley}. We briefly discuss some of these applications, highlighting the wide range of areas where differential geometry plays a crucial role.

\section{Conclusion}
Differential geometry is a fascinating and versatile branch of mathematics with applications in many scientific disciplines. This paper aimed to provide a gentle introduction to some of the fundamental concepts and results in differential geometry. We hope it encourages further exploration into this captivating subject.

\bibliographystyle{plain}
\bibliography{bibliography}

\end{document}
